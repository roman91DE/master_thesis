% Options for packages loaded elsewhere
\PassOptionsToPackage{unicode}{hyperref}
\PassOptionsToPackage{hyphens}{url}
%
\documentclass[
  12pt,
]{article}
\usepackage{amsmath,amssymb}
\usepackage{lmodern}
\usepackage{iftex}
\ifPDFTeX
  \usepackage[T1]{fontenc}
  \usepackage[utf8]{inputenc}
  \usepackage{textcomp} % provide euro and other symbols
\else % if luatex or xetex
  \usepackage{unicode-math}
  \defaultfontfeatures{Scale=MatchLowercase}
  \defaultfontfeatures[\rmfamily]{Ligatures=TeX,Scale=1}
\fi
% Use upquote if available, for straight quotes in verbatim environments
\IfFileExists{upquote.sty}{\usepackage{upquote}}{}
\IfFileExists{microtype.sty}{% use microtype if available
  \usepackage[]{microtype}
  \UseMicrotypeSet[protrusion]{basicmath} % disable protrusion for tt fonts
}{}
\makeatletter
\@ifundefined{KOMAClassName}{% if non-KOMA class
  \IfFileExists{parskip.sty}{%
    \usepackage{parskip}
  }{% else
    \setlength{\parindent}{0pt}
    \setlength{\parskip}{6pt plus 2pt minus 1pt}}
}{% if KOMA class
  \KOMAoptions{parskip=half}}
\makeatother
\usepackage{xcolor}
\usepackage[margin=1in]{geometry}
\usepackage{graphicx}
\makeatletter
\def\maxwidth{\ifdim\Gin@nat@width>\linewidth\linewidth\else\Gin@nat@width\fi}
\def\maxheight{\ifdim\Gin@nat@height>\textheight\textheight\else\Gin@nat@height\fi}
\makeatother
% Scale images if necessary, so that they will not overflow the page
% margins by default, and it is still possible to overwrite the defaults
% using explicit options in \includegraphics[width, height, ...]{}
\setkeys{Gin}{width=\maxwidth,height=\maxheight,keepaspectratio}
% Set default figure placement to htbp
\makeatletter
\def\fps@figure{htbp}
\makeatother
\setlength{\emergencystretch}{3em} % prevent overfull lines
\providecommand{\tightlist}{%
  \setlength{\itemsep}{0pt}\setlength{\parskip}{0pt}}
\setcounter{secnumdepth}{5}
\newlength{\cslhangindent}
\setlength{\cslhangindent}{1.5em}
\newlength{\csllabelwidth}
\setlength{\csllabelwidth}{3em}
\newlength{\cslentryspacingunit} % times entry-spacing
\setlength{\cslentryspacingunit}{\parskip}
\newenvironment{CSLReferences}[2] % #1 hanging-ident, #2 entry spacing
 {% don't indent paragraphs
  \setlength{\parindent}{0pt}
  % turn on hanging indent if param 1 is 1
  \ifodd #1
  \let\oldpar\par
  \def\par{\hangindent=\cslhangindent\oldpar}
  \fi
  % set entry spacing
  \setlength{\parskip}{#2\cslentryspacingunit}
 }%
 {}
\usepackage{calc}
\newcommand{\CSLBlock}[1]{#1\hfill\break}
\newcommand{\CSLLeftMargin}[1]{\parbox[t]{\csllabelwidth}{#1}}
\newcommand{\CSLRightInline}[1]{\parbox[t]{\linewidth - \csllabelwidth}{#1}\break}
\newcommand{\CSLIndent}[1]{\hspace{\cslhangindent}#1}
\usepackage{float}
\usepackage[numbers]{natbib}
\let\origfigure\figure
\let\endorigfigure\endfigure
\renewenvironment{figure}[1][2] {
    \expandafter\origfigure\expandafter[H]
} {
    \endorigfigure
}
\usepackage{float}
\usepackage{booktabs}
\usepackage{longtable}
\usepackage{array}
\usepackage{multirow}
\usepackage{wrapfig}
\usepackage{colortbl}
\usepackage{pdflscape}
\usepackage{tabu}
\usepackage{threeparttable}
\usepackage{threeparttablex}
\usepackage[normalem]{ulem}
\usepackage{makecell}
\usepackage{xcolor}
\ifLuaTeX
  \usepackage{selnolig}  % disable illegal ligatures
\fi
\IfFileExists{bookmark.sty}{\usepackage{bookmark}}{\usepackage{hyperref}}
\IfFileExists{xurl.sty}{\usepackage{xurl}}{} % add URL line breaks if available
\urlstyle{same} % disable monospaced font for URLs
\hypersetup{
  hidelinks,
  pdfcreator={LaTeX via pandoc}}

\title{Pre-Training Denoising Autoencoders for Genetic Programming

(subtitle)

--------------------------------------------------------

Student: Roman Höhn

Date of Birth: 1991-04-14

Place of Birth: Wiesbaden, Hesse

Student ID: 2712497

Supervisor: David Wittenberg

--------------------------------------------------------

Master Thesis

FB 03: Chair of Business Administration and Computer Science

Johannes Gutenberg University Mainz

Summerterm 2022}
\author{}
\date{\vspace{-2.5em}Date of Submission: 2022-09-09}

\begin{document}
\maketitle

\thispagestyle{empty} \newpage{}

\setcounter{page}{1}

\tableofcontents
\thispagestyle{empty}
\newpage

\listoftables
\listoffigures
\thispagestyle{empty}
\newpage

\hypertarget{expose}{%
\section{Expose}\label{expose}}

\hypertarget{introduction}{%
\subsection{Introduction}\label{introduction}}

Denoising Autoencoder Genetic Programming (DAE-GP) is a novel variation
of an genetic programming based Estimation of Distribution Algorithm
(EDA-GP) that uses a denoising autoencoders long short-term memory
network (DAE-LSTMs) as a probabilistic model to sample new candidate
solutions (Wittenberg, Rothlauf and Schweim, 2020).

DAE-LSTMs are recurrent neural networks (RNNs) that can be trained in an
unsupervised learning environment to minimize a reconstruction error for
encoding input data into a compressed representation and subsequently
decoding the compressed representation back to the input dimension. In
DAE-GP, DAE-LSTMs are trained with a subset of high-fitness solutions
selected from the parent population as input. The resulting model is
then used to sample new offspring solutions by propagating partially
mutated solutions from the parent population through the DAE-LSTMS
(Wittenberg, Rothlauf and Schweim, 2020).

In previous work DAE-GP has been demonstrated to outperform GP for both
a generalized version of the royal tree problem (Wittenberg, Rothlauf
and Schweim, 2020) as well as for a standard symbolic regression problem
(Wittenberg and Rothlauf, 2022). The DAE-GP algorithm implemented for
both experiments initializes the DAE-LSTMs with randomized parameters
for each generation \(g\) of the search. A possible way of improving the
quality of solutions and/or the total runtime of DAE-GP could be to
instead use pre-trained DAE-LSTMs that are passed on from each
generation \(g_{t}\) to the next generation \(g_{t+1}\). Pre-Training is
a commonly used strategy in deep architectures that has been shown to
improve both the optimization process itself as well as the
generalization behavior if compared to randomly initialized parameters
(Erhan \emph{et al.}, 2009).

\hypertarget{research-question}{%
\subsection{Research Question}\label{research-question}}

The aim of this research project is to study the effects of using
pre-trained DAE-LSTMs in DAE-GP on both solution quality as well as on
total time consumption.

\ldots what problem(s)???\ldots{}

\hypertarget{research-design}{%
\subsection{Research Design}\label{research-design}}

\newpage

\hypertarget{I}{%
\section*{I - References}\label{I}}
\addcontentsline{toc}{section}{I - References}

\hypertarget{refs}{}
\begin{CSLReferences}{0}{0}
\leavevmode\vadjust pre{\hypertarget{ref-pmlr-v5-erhan09a}{}}%
Erhan, D. \emph{et al.} (2009) {``The difficulty of training deep
architectures and the effect of unsupervised pre-training,''} in D. van
Dyk and M. Welling (eds.) \emph{Proceedings of the twelth international
conference on artificial intelligence and statistics}. Hilton Clearwater
Beach Resort, Clearwater Beach, Florida USA: PMLR (Proceedings of
machine learning research), pp. 153--160. Available at:
\url{https://proceedings.mlr.press/v5/erhan09a.html}.

\leavevmode\vadjust pre{\hypertarget{ref-dae-gp_2022_symreg}{}}%
Wittenberg, D. and Rothlauf, F. (2022) {``Denoising autoencoder genetic
programming for real-world symbolic regression,''} in \emph{Proceedings
of the genetic and evolutionary computation conference companion}. New
York, NY, USA: Association for Computing Machinery (GECCO '22), pp.
612--614. Available at: \url{https://doi.org/10.1145/3520304.3528921}.

\leavevmode\vadjust pre{\hypertarget{ref-dae-gp_2020_rtree}{}}%
Wittenberg, D., Rothlauf, F. and Schweim, D. (2020) {``DAE-GP: Denoising
autoencoder LSTM networks as probabilistic models in estimation of
distribution genetic programming,''} in \emph{Proceedings of the 2020
genetic and evolutionary computation conference}. New York, NY, USA:
Association for Computing Machinery (GECCO '20), pp. 1037--1045.
Available at: \url{https://doi.org/10.1145/3377930.3390180}.

\end{CSLReferences}

\hypertarget{II}{%
\section*{II - Statutory Declaration}\label{II}}
\addcontentsline{toc}{section}{II - Statutory Declaration}

\includegraphics{./private/erklaerung.pdf}\\

\end{document}
