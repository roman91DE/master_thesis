% Options for packages loaded elsewhere
\PassOptionsToPackage{unicode}{hyperref}
\PassOptionsToPackage{hyphens}{url}
%
\documentclass[
  11pt,
]{article}
\usepackage{amsmath,amssymb}
\usepackage{lmodern}
\usepackage{iftex}
\ifPDFTeX
  \usepackage[T1]{fontenc}
  \usepackage[utf8]{inputenc}
  \usepackage{textcomp} % provide euro and other symbols
\else % if luatex or xetex
  \usepackage{unicode-math}
  \defaultfontfeatures{Scale=MatchLowercase}
  \defaultfontfeatures[\rmfamily]{Ligatures=TeX,Scale=1}
\fi
% Use upquote if available, for straight quotes in verbatim environments
\IfFileExists{upquote.sty}{\usepackage{upquote}}{}
\IfFileExists{microtype.sty}{% use microtype if available
  \usepackage[]{microtype}
  \UseMicrotypeSet[protrusion]{basicmath} % disable protrusion for tt fonts
}{}
\makeatletter
\@ifundefined{KOMAClassName}{% if non-KOMA class
  \IfFileExists{parskip.sty}{%
    \usepackage{parskip}
  }{% else
    \setlength{\parindent}{0pt}
    \setlength{\parskip}{6pt plus 2pt minus 1pt}}
}{% if KOMA class
  \KOMAoptions{parskip=half}}
\makeatother
\usepackage{xcolor}
\usepackage[margin=1in]{geometry}
\usepackage{color}
\usepackage{fancyvrb}
\newcommand{\VerbBar}{|}
\newcommand{\VERB}{\Verb[commandchars=\\\{\}]}
\DefineVerbatimEnvironment{Highlighting}{Verbatim}{commandchars=\\\{\}}
% Add ',fontsize=\small' for more characters per line
\usepackage{framed}
\definecolor{shadecolor}{RGB}{248,248,248}
\newenvironment{Shaded}{\begin{snugshade}}{\end{snugshade}}
\newcommand{\AlertTok}[1]{\textcolor[rgb]{0.94,0.16,0.16}{#1}}
\newcommand{\AnnotationTok}[1]{\textcolor[rgb]{0.56,0.35,0.01}{\textbf{\textit{#1}}}}
\newcommand{\AttributeTok}[1]{\textcolor[rgb]{0.77,0.63,0.00}{#1}}
\newcommand{\BaseNTok}[1]{\textcolor[rgb]{0.00,0.00,0.81}{#1}}
\newcommand{\BuiltInTok}[1]{#1}
\newcommand{\CharTok}[1]{\textcolor[rgb]{0.31,0.60,0.02}{#1}}
\newcommand{\CommentTok}[1]{\textcolor[rgb]{0.56,0.35,0.01}{\textit{#1}}}
\newcommand{\CommentVarTok}[1]{\textcolor[rgb]{0.56,0.35,0.01}{\textbf{\textit{#1}}}}
\newcommand{\ConstantTok}[1]{\textcolor[rgb]{0.00,0.00,0.00}{#1}}
\newcommand{\ControlFlowTok}[1]{\textcolor[rgb]{0.13,0.29,0.53}{\textbf{#1}}}
\newcommand{\DataTypeTok}[1]{\textcolor[rgb]{0.13,0.29,0.53}{#1}}
\newcommand{\DecValTok}[1]{\textcolor[rgb]{0.00,0.00,0.81}{#1}}
\newcommand{\DocumentationTok}[1]{\textcolor[rgb]{0.56,0.35,0.01}{\textbf{\textit{#1}}}}
\newcommand{\ErrorTok}[1]{\textcolor[rgb]{0.64,0.00,0.00}{\textbf{#1}}}
\newcommand{\ExtensionTok}[1]{#1}
\newcommand{\FloatTok}[1]{\textcolor[rgb]{0.00,0.00,0.81}{#1}}
\newcommand{\FunctionTok}[1]{\textcolor[rgb]{0.00,0.00,0.00}{#1}}
\newcommand{\ImportTok}[1]{#1}
\newcommand{\InformationTok}[1]{\textcolor[rgb]{0.56,0.35,0.01}{\textbf{\textit{#1}}}}
\newcommand{\KeywordTok}[1]{\textcolor[rgb]{0.13,0.29,0.53}{\textbf{#1}}}
\newcommand{\NormalTok}[1]{#1}
\newcommand{\OperatorTok}[1]{\textcolor[rgb]{0.81,0.36,0.00}{\textbf{#1}}}
\newcommand{\OtherTok}[1]{\textcolor[rgb]{0.56,0.35,0.01}{#1}}
\newcommand{\PreprocessorTok}[1]{\textcolor[rgb]{0.56,0.35,0.01}{\textit{#1}}}
\newcommand{\RegionMarkerTok}[1]{#1}
\newcommand{\SpecialCharTok}[1]{\textcolor[rgb]{0.00,0.00,0.00}{#1}}
\newcommand{\SpecialStringTok}[1]{\textcolor[rgb]{0.31,0.60,0.02}{#1}}
\newcommand{\StringTok}[1]{\textcolor[rgb]{0.31,0.60,0.02}{#1}}
\newcommand{\VariableTok}[1]{\textcolor[rgb]{0.00,0.00,0.00}{#1}}
\newcommand{\VerbatimStringTok}[1]{\textcolor[rgb]{0.31,0.60,0.02}{#1}}
\newcommand{\WarningTok}[1]{\textcolor[rgb]{0.56,0.35,0.01}{\textbf{\textit{#1}}}}
\usepackage{graphicx}
\makeatletter
\def\maxwidth{\ifdim\Gin@nat@width>\linewidth\linewidth\else\Gin@nat@width\fi}
\def\maxheight{\ifdim\Gin@nat@height>\textheight\textheight\else\Gin@nat@height\fi}
\makeatother
% Scale images if necessary, so that they will not overflow the page
% margins by default, and it is still possible to overwrite the defaults
% using explicit options in \includegraphics[width, height, ...]{}
\setkeys{Gin}{width=\maxwidth,height=\maxheight,keepaspectratio}
% Set default figure placement to htbp
\makeatletter
\def\fps@figure{htbp}
\makeatother
\setlength{\emergencystretch}{3em} % prevent overfull lines
\providecommand{\tightlist}{%
  \setlength{\itemsep}{0pt}\setlength{\parskip}{0pt}}
\setcounter{secnumdepth}{5}
\newlength{\cslhangindent}
\setlength{\cslhangindent}{1.5em}
\newlength{\csllabelwidth}
\setlength{\csllabelwidth}{3em}
\newlength{\cslentryspacingunit} % times entry-spacing
\setlength{\cslentryspacingunit}{\parskip}
\newenvironment{CSLReferences}[2] % #1 hanging-ident, #2 entry spacing
 {% don't indent paragraphs
  \setlength{\parindent}{0pt}
  % turn on hanging indent if param 1 is 1
  \ifodd #1
  \let\oldpar\par
  \def\par{\hangindent=\cslhangindent\oldpar}
  \fi
  % set entry spacing
  \setlength{\parskip}{#2\cslentryspacingunit}
 }%
 {}
\usepackage{calc}
\newcommand{\CSLBlock}[1]{#1\hfill\break}
\newcommand{\CSLLeftMargin}[1]{\parbox[t]{\csllabelwidth}{#1}}
\newcommand{\CSLRightInline}[1]{\parbox[t]{\linewidth - \csllabelwidth}{#1}\break}
\newcommand{\CSLIndent}[1]{\hspace{\cslhangindent}#1}
\usepackage{float}
\usepackage[numbers]{natbib}
\let\origfigure\figure
\let\endorigfigure\endfigure
\renewenvironment{figure}[1][2] {
    \expandafter\origfigure\expandafter[H]
} {
    \endorigfigure
}
\usepackage{float}
\usepackage{booktabs}
\usepackage{longtable}
\usepackage{array}
\usepackage{multirow}
\usepackage{wrapfig}
\usepackage{colortbl}
\usepackage{pdflscape}
\usepackage{tabu}
\usepackage{threeparttable}
\usepackage{threeparttablex}
\usepackage[normalem]{ulem}
\usepackage{makecell}
\usepackage{xcolor}
\ifLuaTeX
  \usepackage{selnolig}  % disable illegal ligatures
\fi
\IfFileExists{bookmark.sty}{\usepackage{bookmark}}{\usepackage{hyperref}}
\IfFileExists{xurl.sty}{\usepackage{xurl}}{} % add URL line breaks if available
\urlstyle{same} % disable monospaced font for URLs
\hypersetup{
  hidelinks,
  pdfcreator={LaTeX via pandoc}}

\title{Pre-Trained Denoising Autoencoders Long Short-Term Memory
Networks as probabilistic Models for Estimation of Distribution Genetic
Programming

--------------------------------------------------------

Student: Roman Höhn

Date of Birth: 1991-04-14

Place of Birth: Wiesbaden, Hesse

Student ID: 2712497

Supervisor: David Wittenberg

--------------------------------------------------------

Master Thesis

FB 03: Chair of Business Administration and Computer Science

Johannes Gutenberg University Mainz}
\author{}
\date{\vspace{-2.5em}Date of Submission: 2022-12-23}

\begin{document}
\maketitle

\thispagestyle{empty} \newpage{}

\setcounter{page}{1}
\tableofcontents
\thispagestyle{empty}
\newpage  
\listoftables
\listoffigures
\thispagestyle{empty}
\newpage

\hypertarget{abstract}{%
\section{Abstract}\label{abstract}}

\ldots{}

\hypertarget{introduction}{%
\section{Introduction}\label{introduction}}

Denoising Autoencoder Genetic Programming (DAE-GP) is a novel variation
of an genetic programming based Estimation of Distribution Algorithm
(EDA-GP) that uses a denoising autoencoders long short-term memory
network (DAE-LSTMs) as a probabilistic model to sample new candidate
solutions {[}18{]}.

DAE-LSTMs are recurrent neural networks (RNNs) that can be trained in an
unsupervised learning environment to minimize a reconstruction error for
encoding input data into a compressed representation and subsequently
decoding the compressed representation back to the input dimension. In
DAE-GP, DAE-LSTMs are trained with a subset of high-fitness solutions
selected from a parent population with the aim to capture their
promising qualities. The resulting model is then used to sample new
offspring solutions by propagating partially mutated solutions from the
parent population through the DAE-LSTMS {[}18{]}. In previous work
DAE-GP has been shown to outperform GP for both a generalized version of
the royal tree problem {[}18{]} as well as for a real-world symbolic
regression problem {[}17{]}.

The DAE-GP algorithm first described by {[}18{]} trains a DAE-LSTMs for
each generation \(g\) of the search from scratch. {[}17{]} and {[}16{]}
suggests the incorporation of a pre-training strategy into the
evolutionary search as a possible way of improving the model quality of
DAE-GP. The key idea is to pre-train an initial DAE-LSTM on a large
population of candidate solutions and to use the weights of this initial
model in each generation of the search as a starting point. This thesis
studies the influence of using pre-trained DAE-LSTMs in DAE-GP for
symbolic regression, especially looking at the influence in
generalization behavior and the overall quality of solutions found by
DAE-GP. The aim of this study is to answer the question if a
pre-training strategy can help to improve generalization in DAE-GP and
if so, analyze how this improved generalization ability benefits the
overall performance of DAE-GP.

\ldots{}

\hypertarget{theoretical-foundations}{%
\section{Theoretical Foundations}\label{theoretical-foundations}}

\emph{This section describes the relevant concepts that are necessary
for the understanding and classification of DAE-GP as well as the
concept of pre-training in artificial neural networks}

\hypertarget{denoising-autoencoder-genetic-programming}{%
\subsection{Denoising Autoencoder Genetic
Programming}\label{denoising-autoencoder-genetic-programming}}

DAE-GP is an EDA-GP algorithm that uses DAE-LSTM networks as a
probabilistic model to sample new offspring solutions {[}18{]}.

\hypertarget{evolutionary-computation}{%
\subsubsection{Evolutionary
Computation}\label{evolutionary-computation}}

As a variant of GP, DAE-GP can be classified as a meta-heuristic that
belongs to the field of evolutionary computation (EC). EC based
meta-heuristics are optimization methods that simulate the process of
Darwinian evolution to search for high quality solutions by applying
selection and variation operators to a population of candidate
solutions. Examples of EC include genetic algorithms (GA), evolutionary
strategies (ES) and GP. In EC, the quality of a solution is commonly
measured as fitness and the time steps of the search are called
generations. Another important concept in EC is the distinction between
genotypes and phenotypes of solutions, the genotype contains the
information that is necessary to construct the phenotype, the outer
appearance of a particular solution on which we measure the overall
quality of solutions. The representation of a solution is therefore
defined by the mapping of genotypes to phenotypes {[}12{]}. Genetic
operators, such as mutation or recombination are usually applied to the
genotype of solutions.

\hypertarget{genetic-programming}{%
\subsubsection{Genetic Programming}\label{genetic-programming}}

GP follows the same basic evolutionary principle of EC but searches for
more general, hierarchical computer programs of dynamically varying size
and shape {[}8{]}. The computer programs that are at the center of the
evolutionary search in GP are commonly represented by tree structures at
the level of their phenotype {[}12{]}. Since GP searches for high
fitness computer programs that produce a desired output for some input,
it can be applied to various different problem domains such as symbolic
regression, automatic programming, or evolving game-playing strategies
{[}8{]}. An important quality of GP is the ability to search for
solutions of variable length and structure. GP is an especially useful
meta heuristic for problems where no a priori knowledge about the final
form of good solutions is available. GPs ability to optimize solutions
for their structure as well as for their parameters led to it being one
of the most prevalent methods used symbolic regression {[}10{]}.

\hypertarget{estimation-of-distribution-algorithms}{%
\subsubsection{Estimation of Distribution
Algorithms}\label{estimation-of-distribution-algorithms}}

The aim of Estimation of distribution algorithms (EDA) is to replace the
standard variation operators used in GA by building probabilistic models
that can capture complex dependencies between different decision
variables of an optimization problem {[}12{]}. EDAs use this
probabilistic model to sample new offspring solutions inside an
evolutionary search to replace crossover and/or mutation operators.

\hypertarget{denoising-autoencoders}{%
\subsubsection{Denoising Autoencoders}\label{denoising-autoencoders}}

One possible way of model building in EDA proposed by {[}11{]} is to use
denoising autoencoders (DAEs) as generative models to capture complex
probability distributions of decision variables.

DAE, a variation of the autoencoders (AE), are a widely used type of
neural networks in the domain of unsupervized machine learning that maps
\(n\) input variables to \(n\) output variables using a hidden
representation.

AE were introduced by {[}6{]} to compress high-dimensional data into
lower-dimensions. An AE consists of two different subunits:

\begin{itemize}
\tightlist
\item
  Encoder \(g(x)\): Encode input data to a smaller cental layer \(h\)
\item
  Decoder \(d(h)\): Decode and output the the encoded data back to its
  original dimension
\end{itemize}

The AE is trained to reduce the reconstruction error between input and
output data, after the training procedure is finished the network is
able to reduce the dimensionality of input data to a compressed
representation{[}6{]}.

DAE was first introduced by {[}14{]} as an improved AE with the ability
to learn new representations of data that is especially robust to
partially corrupted input data. DAE modifies the AE by using partially
corrupted input data for the AE and training it to reconstruct the
uncorrupted, original version of the input data.

Since the hidden representation of DAE captures the dependency structure
of the input data it can therefore also be used to generate new
solutions in the context of GAs {[}11{]}.

DAE-GP builds upon the concept of using DAEs in EDAs described by
{[}11{]} and transfers the concept to the domain of GP. The mutation and
crossover operators of standard GP are replaced by sampling new
solutions from a probabilistic model that is build by training a DAEs
long short-term memory (LSTM) network on a subset of high fitness
solutions from the current population {[}18{]}. LSTMs are a variant of
recurrent neural networks first introduced by {[}7{]} that can store
learned information over en extended time period while avoiding the
problem of vanishing and/or exploding gradients. Since DAE-GP encodes
candidate solutions as linear strings in prefix notation, the DAE in
DAE-GP uses LSTMs for both encoding and decoding where the total amount
of time steps \(T\) is equal to sum of the length of the input solution
and the output solution {[}18{]}.

\hypertarget{pre-training}{%
\subsection{Pre-Training}\label{pre-training}}

The strategy of pre-training artificial neural networks is based on the
idea of transfer learning. Transfer learning in the field of machine
learning aims at retaining previously learned knowledge from one task to
re-use it for another task {[}5{]} {[}9{]}. Transfer learning
traditionally follows a two phase approach:

\begin{enumerate}
\def\labelenumi{\arabic{enumi}.}
\tightlist
\item
  Pre-Training Phase: Capture knowledge from source task
\item
  Fine-Tuning Phase: Transfer knowledge to the target task
\end{enumerate}

where source task and target task are usually similar but may differ in
their feature space and the distribution of training data {[}9{]}. One
of the main motivations for using transfer learning in real world
machine learning tasks is the ability to reduce the need for large
amounts of training data which can often be unavailable or too expensive
to collect.

Pre-Training is a commonly used strategy in deep architectures that has
been shown to improve both the optimization process itself as well as
the generalization behavior if compared to randomly initialized
parameters {[}3{]}.

\ldots describe different pre-training strategies, e.g.~few shot,
perpetual, re-using\ldots{}

-- describe unsupervized vs supervized pre-training\ldots{}

\hypertarget{pre-training-in-dae-gp}{%
\section{Pre-Training in DAE-GP}\label{pre-training-in-dae-gp}}

The pre-training strategy used in this thesis is applied to the DAE-LSTM
model \(M_g\) that are used in each DAE-GP generation \(g\) for
\(g\in{1,...,g_{max}}\). Instead of initializing and optimizing each
model from scratch as done in previous work (e.g. {[}18{]} {[}17{]}), a
separate DAE-LSTM network \(\hat{M}\) will be trained on a large
population of randomly initialized solutions.\\
The optimal parameters obtained after finishing the training procedure
of \(\hat{M}\) are then used as the starting parameters for each
following DAE-LSTM model \(M_g\) for \(g\in{1,...,g_{max}}\).

The motivation for incorporating pre-training into DAE-GP is based on
the following suspected mechanisms of improvement:

\begin{enumerate}
\def\labelenumi{\arabic{enumi}.}
\tightlist
\item
  Improve overall performance of DAE-GP by improving either run time
  and/or solution quality
\item
  Reduce the need for large population sizes to avoid sampling error
\item
  Improve model robustness and generalization ability
\end{enumerate}

The probably most obvious motivation for using a pre-training strategy
in DAE-GP is based on the fact, that initializing \(M_g\) with
pre-trained weights from \(\hat{M}\) is likely to reduce the amount of
training epochs that are necessary until the point of training error
convergence is reached.

Early research on pre-training for DAE by {[}3{]} comes to the
conclusion that besides adding robustness to models the strategy also
results in improved generalization and better performing models. The
authors describe that unsupervised pre-training behaves like a regulizer
on DAE networks, the mean test error and it's variance are reduced for
DAE networks that are initialized from pre-trained weights if compared
to the same architectures that use random initialization.

Another important finding by {[}3{]} is that the positive effect of
pre-training described above is dependent on both the depth of the
network as well as the size of layers - while increasingly larger
networks benefit increasingly more from pre-training the final
performance of small architectures tends to be worse with pre-training
than with randomized weight initialization. The authors find evidence
that pre-training DAE is especially usefull for optimizing the
parameters of the lower layers of the network.

Another motivation for introducing pre-training into DAE-GP is the
prevalence of sampling error for small GP populations sizes. {[}13{]}
finds that sampling error, non-systematic errors that are caused by
observing only a small subset of the statistical population, is a severe
problem in the domain of GP since the initial population may not be a
representative sample of all possible solutions. {[}13{]} also
introduces a method for calculating optimal population sizes to minimize
the presence of sampling error. Since DAE-LSTM of DAE-GP learns the
properties of it's training population and reuses the acquired knowledge
in the sampling procedure, using the parameters obtained from training
\(\hat{M}\) on a sufficiently large training population might reduce the
need for large population sizes in the following generations of the
search if \(\hat{M}\) already implicitly captured the properties of a
representative sample of solutions. If this mechanism can be applied
successfully it would benefit the performance of DAE-GP by increasing
the population diversity (resulting in better solution quality) as well
as by reducing the need for large population sizes which require
computational resources.

\hypertarget{implementation}{%
\subsection{Implementation}\label{implementation}}

The DAE-GP algorithm, first described by {[}18{]}, is described by
algorithm 1

\begin{Shaded}
\begin{Highlighting}[numbers=left,,]
\NormalTok{WHILE NOT termination\_criterion}
\NormalTok{  DO }
\NormalTok{END WHILE}
\NormalTok{RETURN}
\end{Highlighting}
\end{Shaded}

One difficulty in the implementation of a pre-training strategy into
DAE-GP has been the detemination of the number of hidden neurons in side
the DAE-LSTMs hidden layers. The original description of DAE-GP {[}18{]}
used a strategy where the number of hidden neurons for each hidden layer
is dynamically set per generation to the maximum individual size inside
the current population. For a pre-training implementation, this strategy
can not be easily adapted since it leads to a changing number of neurons
at each generation resulting in different dimensions of the DAE-LSTM. To
allow for the sharing of weights and biases from the pre-trained model
\(\hat{M}\) to each \(M_g\) for \(g\in{1,...,g_{max}}\) I experimented
with two different strategies:

\begin{enumerate}
\def\labelenumi{\arabic{enumi}.}
\tightlist
\item
  Set the number of hidden neurons per hidden layer for each generation
  to the maximum individual size inside the pre-training population
\item
  (dependent on the terminal/function set)
\end{enumerate}

The framework that was used to conduct all experiments of this thesis
was provided by the supervisor David Wittenberg and uses the python
programming language in conjunction with the baseline libraries
\texttt{Keras} {[}1{]} for deep neural networks and \texttt{deap}
{[}4{]} for evolutionary computation.

\hypertarget{benchmark-problem}{%
\subsection{Benchmark Problem}\label{benchmark-problem}}

To test pre-training in DAE-GP this thesis focuses on the domain of
real-world symbolic regression problems. Symbolic Regression problems
have been one of the first GP applications {[}8{]} and are an actively
studied and highly relevant research area. The goal in symbolic
regression is to find a mathematical model for a given set of data
points {[}10{]}, in real-world symbolic regression these data points are
sourced from real-world observations which in contrast to synthetic
symbolic regression problem are more likely to contain random noise and
bias. Another important challenge in solving real-world symbolic
regression problems is the ability for a given model to generalize, we
want the final model to show high accuracy in predicting outcomes for
previously unseen cases.

The main experiments conducted in this thesis uses the NASA Airfoil
Self-Noise Data Set which is part of the UCI machine learning repository
{[}2{]}. The dataset consists of 5 input variables and 1 output variable
that are listed in table 1.

\begin{table}[!h]

\caption{\label{tab:airfoil_dataset_description}Airfoil - Dataset Description}
\centering
\begin{tabular}[t]{l|l|l|l}
\hline
\textbf{Type} & \textbf{Name} & \textbf{Description} & \textbf{Unit}\\
\hline
input & x1 & Frequency & Hertz\\
\hline
input & x2 & Angle of attack & Degree\\
\hline
input & x3 & Chord length & meters\\
\hline
input & x4 & Free-stream velocity & meters/second\\
\hline
input & x5 & Suction side displacement thickness & meters\\
\hline
output & y & Scaled sound pressure level & decibels\\
\hline
\end{tabular}
\end{table}

The objective of the airfoil problem is to find a function that
accurately predicts the output variable \(y\) by taking in a subset of
the input variables \(x1,x2,x3,x4,x5\). The function set used by all
DAE-GP variations for the airfoil problem is summarized in table 2, the
terminal set consists of the 5 input variables \(x1,x2,x3,x4,x5\) and
ephemeral random integers in the range of \([{}-5,..,5]\) {[}17{]}.

\begin{table}[!h]

\caption{\label{tab:airfoil_function_set}Airfoil - Function Set}
\centering
\begin{tabular}[t]{l|r}
\hline
\textbf{function.} & \textbf{arity}\\
\hline
addition & 2\\
\hline
subtraction & 2\\
\hline
multiplication & 2\\
\hline
analytic\_quotient & 2\\
\hline
\end{tabular}
\end{table}

\hypertarget{results}{%
\section{Results}\label{results}}

\hypertarget{effect-on-generalisation}{%
\subsection{Effect on Generalisation}\label{effect-on-generalisation}}

To gather a deeper understanding about the effect of pre-training
DAE-LSTM in DAE-GP a series of experiments was conducted using the
airfoil dataset for symbolic regression while using and different
parameter configurations for the number of hidden layers as well as the
number of hidden neurons per hidden layer.

To study generalization behavior this series of experiments is conducted
using DAE-GP with only a single generation until the search terminates.
I disregard the fitness of solutions found and focus soley on the
reconstruction error that is produced during the training of each the
DAE-LSTM. The reconstruction error is measured for two separate
populations, a training population \(train\_pop\) that is used to train
our DAE-LSTM as well as a hold-out validation population \(test\_pop\).
For the pre-trained DAE-GP two additional, separate populations
\(\hat{train\_pop}\) and \(\hat{test\_pop}\) are used only for
pre-training.

DAE-GP is tested in two different configurations:

\begin{itemize}
\tightlist
\item
  Variable number of hidden neurons (50, 100, 200) with a single hidden
  layer
\item
  Fixed number of hidden neurons (100) per hidden layer with variable
  number of hidden layers (1, 2, 3)
\end{itemize}

For each configuration I tested traditional DAE-GP as well as a
pre-trained DAE-GP resulting in 12 total sub experiments that were each
based on 10 individual runs (total number of runs=\(120\)). To avoid
creating biased results through the presence of sampling error inside
the pre-training population (see {[}13{]}), the population size for the
pre-training phase is chosen very high with \(\hat{pop\_size}=20000\)
where 50\% of \(\hat{pop}\) is used for the training population
\(\hat{train\_pop}\) and the remaining 50\% are used for the hold-out
validation population \(\hat{test\_pop}\). The training of DAE-LSTM uses
a fixed number of 1000 epochs until termination and uses Adam
optimization for gradient descent. The reason for using a high amount of
1000 fixed training epochs is to deliberately force the DAE-LSTM to
overfit to the training data.

In general I expect that with a growing number of model parameters
(either by adding more hidden layers or more hidden neurons per layer)
the DAE-LSTM will be more prone to overfit to the training population
\(train\_pop\) resulting in a small reconstruction error for the
training population and a large one for the validation population
\(train\_pop\). The reason for this effect is that a larger network
trained over an extended period of time (without the use of strategies
like early stopping), has much more potential to learn noise from the
training dataset than a smaller network, which is more likely to result
in worsening performance for previously unseen cases {[}15{]}.

Based on the review of {[}3{]} I also expect that pre-training will have
an insignificant or even negative influence on small DAE-LSTM instances
while improving the networks generalization ability with growing size.

\begin{figure}
\centering
\includegraphics{./img/airfoil_firstGen/airfoil_firstGen_median_training_error_by_neurons.png}
\caption{Airfoil - First Generation Median Training Error for variable
number of hidden Neurons}
\end{figure}

\begin{figure}
\centering
\includegraphics{./img/airfoil_firstGen/airfoil_firstGen_boxplot_training_error_by_neurons.png}
\caption{Airfoil - First Generation final Training Error for variable
number of hidden Neurons}
\end{figure}

\begin{figure}
\centering
\includegraphics{./img/airfoil_firstGen/airfoil_firstGen_mean_training_error_by_layers.png}
\caption{Airfoil - First Generation Median Training Error for variable
number of hidden Layers}
\end{figure}

\begin{figure}
\centering
\includegraphics{./img/airfoil_firstGen/airfoil_firstGen_boxplot_training_error_by_layer.png}
\caption{Airfoil - First Generation final Training Error for variable
number of hidden Layers}
\end{figure}

\hypertarget{effect-on-search-performance}{%
\subsection{Effect on search
performance}\label{effect-on-search-performance}}

After closely examining the effect of pre-training on DAE-GPs
generalization behaviour another series of experiments is conducted to
study how pre-training influences the overall search behaviour of
DAE-GP. Again the Airfoil problem was selected as a first benchmark
problem to test pre-trained DAE-GP against traditional DAE-GP
{[}\textbf{gp\_2022\_symreg?}{]}, the parameters are described in table
\(A\)

\begin{table}[!h]

\caption{\label{tab:airfoil_fullRun_2hl_maxIndSize_params}Airfoil - Parameter for full Run}
\centering
\begin{tabular}[t]{l|l}
\hline
\textbf{parameter} & \textbf{value}\\
\hline
populationSize & 500\\
\hline
generations & 30\\
\hline
fitness & RMSE\\
\hline
TrainingMode & Convergence\\
\hline
SamplingSteps & 2\\
\hline
hiddenLayers & 2\\
\hline
Selection & Binary Tournament Selection\\
\hline
Pre-Training PopulationSize(Training/Validation) & 10000/100000\\
\hline
Pre-Training TrainingMode & Early Stopping\\
\hline
\end{tabular}
\end{table}

Regarding the hidden neurons, the dimension of each hidden layer, I used
two different strategies: Regular DAE-GP uses the same strategy as
described by earlier work (see {[}\textbf{gp\_2022\_symreg?}{]} or
{[}18{]}), the number of hidden neurons is dynamically set at each
generation to the maximum length of all solutions inside the current
training population. For pre-trained DAE-GP another strategy had to be
used, we use a fixed number of hidden neurons for each run that is set
equal to the maximum length of all individuals that are present in the
initial pre-training population.

\hypertarget{discussion}{%
\section{Discussion}\label{discussion}}

\ldots{}

\hypertarget{limitations-and-open-questions}{%
\section{Limitations and open
Questions}\label{limitations-and-open-questions}}

\hypertarget{further-questions}{%
\subsection{Further Questions}\label{further-questions}}

{[}3{]}: Pre-Training effect especially usefull for lower-level layers
-\textgreater{} Only adapt weights from low layers or embedd the
pre-trained model?

Different pre-training strategies?

\newpage

\hypertarget{I}{%
\section*{References}\label{I}}
\addcontentsline{toc}{section}{References}

\hypertarget{refs}{}
\begin{CSLReferences}{0}{0}
\leavevmode\vadjust pre{\hypertarget{ref-chollet2015keras}{}}%
\CSLLeftMargin{{[}1{]} }%
\CSLRightInline{Francois Chollet and others. 2015. Keras. Retrieved from
\url{https://github.com/fchollet/keras}}

\leavevmode\vadjust pre{\hypertarget{ref-machine_learning_repo}{}}%
\CSLLeftMargin{{[}2{]} }%
\CSLRightInline{Dheeru Dua and Casey Graff. 2017. {UCI} machine learning
repository. Retrieved from \url{http://archive.ics.uci.edu/ml}}

\leavevmode\vadjust pre{\hypertarget{ref-pmlr-v5-erhan09a}{}}%
\CSLLeftMargin{{[}3{]} }%
\CSLRightInline{Dumitru Erhan, Pierre-Antoine Manzagol, Yoshua Bengio,
Samy Bengio, and Pascal Vincent. 2009. The difficulty of training deep
architectures and the effect of unsupervised pre-training. In
\emph{Proceedings of the twelth international conference on artificial
intelligence and statistics} (Proceedings of machine learning research),
PMLR, Hilton Clearwater Beach Resort, Clearwater Beach, Florida USA,
153--160. Retrieved from
\url{https://proceedings.mlr.press/v5/erhan09a.html}}

\leavevmode\vadjust pre{\hypertarget{ref-DEAP_JMLR2012}{}}%
\CSLLeftMargin{{[}4{]} }%
\CSLRightInline{Félix-Antoine Fortin, François-Michel De Rainville,
Marc-André Gardner, Marc Parizeau, and Christian Gagné. 2012. {DEAP}:
Evolutionary algorithms made easy. \emph{Journal of Machine Learning
Research} 13, (2012), 2171--2175.}

\leavevmode\vadjust pre{\hypertarget{ref-HAN2021225}{}}%
\CSLLeftMargin{{[}5{]} }%
\CSLRightInline{Xu Han, Zhengyan Zhang, Ning Ding, Yuxian Gu, Xiao Liu,
Yuqi Huo, Jiezhong Qiu, Yuan Yao, Ao Zhang, Liang Zhang, Wentao Han,
Minlie Huang, Qin Jin, Yanyan Lan, Yang Liu, Zhiyuan Liu, Zhiwu Lu,
Xipeng Qiu, Ruihua Song, Jie Tang, Ji-Rong Wen, Jinhui Yuan, Wayne Xin
Zhao, and Jun Zhu. 2021. Pre-trained models: Past, present and future.
\emph{AI Open} 2, (2021), 225--250.
DOI:https://doi.org/\url{https://doi.org/10.1016/j.aiopen.2021.08.002}}

\leavevmode\vadjust pre{\hypertarget{ref-ae_orig}{}}%
\CSLLeftMargin{{[}6{]} }%
\CSLRightInline{G. E. Hinton and R. R. Salakhutdinov. 2006. Reducing the
dimensionality of data with neural networks. \emph{Science} 313, 5786
(2006), 504--507.
DOI:https://doi.org/\href{https://doi.org/10.1126/science.1127647}{10.1126/science.1127647}}

\leavevmode\vadjust pre{\hypertarget{ref-lstm_orig}{}}%
\CSLLeftMargin{{[}7{]} }%
\CSLRightInline{Sepp Hochreiter and Jürgen Schmidhuber. 1997. Long
short-term memory. \emph{Neural computation} 9, (December 1997),
1735--80.
DOI:https://doi.org/\href{https://doi.org/10.1162/neco.1997.9.8.1735}{10.1162/neco.1997.9.8.1735}}

\leavevmode\vadjust pre{\hypertarget{ref-Koza1993GeneticP}{}}%
\CSLLeftMargin{{[}8{]} }%
\CSLRightInline{John R. Koza. 1993. Genetic programming - on the
programming of computers by means of natural selection. In \emph{Complex
adaptive systems}.}

\leavevmode\vadjust pre{\hypertarget{ref-survey_transfer_learning}{}}%
\CSLLeftMargin{{[}9{]} }%
\CSLRightInline{Sinno Jialin Pan and Qiang Yang. 2010. A survey on
transfer learning. \emph{IEEE Transactions on Knowledge and Data
Engineering} 22, 10 (2010), 1345--1359.
DOI:https://doi.org/\href{https://doi.org/10.1109/TKDE.2009.191}{10.1109/TKDE.2009.191}}

\leavevmode\vadjust pre{\hypertarget{ref-10.1007ux2f978-3-540-24621-3_22}{}}%
\CSLLeftMargin{{[}10{]} }%
\CSLRightInline{Grégory Paris, Denis Robilliard, and Cyril Fonlupt.
2004. Exploring overfitting in genetic programming. In \emph{Artificial
evolution}, Springer Berlin Heidelberg, Berlin, Heidelberg, 267--277.}

\leavevmode\vadjust pre{\hypertarget{ref-harmless_overfitting_eda}{}}%
\CSLLeftMargin{{[}11{]} }%
\CSLRightInline{Malte Probst and Franz Rothlauf. 2020. Harmless
overfitting: Using denoising autoencoders in estimation of distribution
algorithms. \emph{Journal of Machine Learning Research} 21, 78 (2020),
1--31. Retrieved from \url{http://jmlr.org/papers/v21/16-543.html}}

\leavevmode\vadjust pre{\hypertarget{ref-design_of_modern_heuristics}{}}%
\CSLLeftMargin{{[}12{]} }%
\CSLRightInline{Franz Rothlauf. 2011. \emph{Design of modern heuristics:
Principles and application}.
DOI:https://doi.org/\href{https://doi.org/10.1007/978-3-540-72962-4}{10.1007/978-3-540-72962-4}}

\leavevmode\vadjust pre{\hypertarget{ref-sampling_err_gp}{}}%
\CSLLeftMargin{{[}13{]} }%
\CSLRightInline{Dirk Schweim, David Wittenberg, and Franz Rothlauf.
2021. On sampling error in genetic programming. \emph{Natural computing}
2021, (2021).
DOI:https://doi.org/\url{http://doi.org/10.25358/openscience-5820}}

\leavevmode\vadjust pre{\hypertarget{ref-10.1145ux2f1390156.1390294}{}}%
\CSLLeftMargin{{[}14{]} }%
\CSLRightInline{Pascal Vincent, Hugo Larochelle, Yoshua Bengio, and
Pierre-Antoine Manzagol. 2008. Extracting and composing robust features
with denoising autoencoders. In \emph{Proceedings of the 25th
international conference on machine learning} (ICML '08), Association
for Computing Machinery, New York, NY, USA, 1096--1103.
DOI:https://doi.org/\href{https://doi.org/10.1145/1390156.1390294}{10.1145/1390156.1390294}}

\leavevmode\vadjust pre{\hypertarget{ref-weigend1994overfitting}{}}%
\CSLLeftMargin{{[}15{]} }%
\CSLRightInline{Andreas Weigend. 1994. On overfitting and the effective
number of hidden units. In \emph{Proceedings of the 1993 connectionist
models summer school}, 335--342.}

\leavevmode\vadjust pre{\hypertarget{ref-daegp_explore_exploit}{}}%
\CSLLeftMargin{{[}16{]} }%
\CSLRightInline{David Wittenberg. 2022. Using denoising autoencoder
genetic programming to~control exploration and~exploitation in~search.
In \emph{Genetic programming}, Springer International Publishing, Cham,
102--117.}

\leavevmode\vadjust pre{\hypertarget{ref-dae-gp_2022_symreg}{}}%
\CSLLeftMargin{{[}17{]} }%
\CSLRightInline{David Wittenberg and Franz Rothlauf. 2022. Denoising
autoencoder genetic programming for real-world symbolic regression. In
\emph{Proceedings of the genetic and evolutionary computation conference
companion} (GECCO '22), Association for Computing Machinery, New York,
NY, USA, 612--614.
DOI:https://doi.org/\href{https://doi.org/10.1145/3520304.3528921}{10.1145/3520304.3528921}}

\leavevmode\vadjust pre{\hypertarget{ref-dae-gp_2020_rtree}{}}%
\CSLLeftMargin{{[}18{]} }%
\CSLRightInline{David Wittenberg, Franz Rothlauf, and Dirk Schweim.
2020. DAE-GP: Denoising autoencoder LSTM networks as probabilistic
models in estimation of distribution genetic programming. In
\emph{Proceedings of the 2020 genetic and evolutionary computation
conference} (GECCO '20), Association for Computing Machinery, New York,
NY, USA, 1037--1045.
DOI:https://doi.org/\href{https://doi.org/10.1145/3377930.3390180}{10.1145/3377930.3390180}}

\end{CSLReferences}

\end{document}
