% Options for packages loaded elsewhere
\PassOptionsToPackage{unicode}{hyperref}
\PassOptionsToPackage{hyphens}{url}
%
\documentclass[
  ignorenonframetext,
]{beamer}
\usepackage{pgfpages}
\setbeamertemplate{caption}[numbered]
\setbeamertemplate{caption label separator}{: }
\setbeamercolor{caption name}{fg=normal text.fg}
\beamertemplatenavigationsymbolsempty
% Prevent slide breaks in the middle of a paragraph
\widowpenalties 1 10000
\raggedbottom
\setbeamertemplate{part page}{
  \centering
  \begin{beamercolorbox}[sep=16pt,center]{part title}
    \usebeamerfont{part title}\insertpart\par
  \end{beamercolorbox}
}
\setbeamertemplate{section page}{
  \centering
  \begin{beamercolorbox}[sep=12pt,center]{part title}
    \usebeamerfont{section title}\insertsection\par
  \end{beamercolorbox}
}
\setbeamertemplate{subsection page}{
  \centering
  \begin{beamercolorbox}[sep=8pt,center]{part title}
    \usebeamerfont{subsection title}\insertsubsection\par
  \end{beamercolorbox}
}
\AtBeginPart{
  \frame{\partpage}
}
\AtBeginSection{
  \ifbibliography
  \else
    \frame{\sectionpage}
  \fi
}
\AtBeginSubsection{
  \frame{\subsectionpage}
}
\usepackage{amsmath,amssymb}
\usepackage{lmodern}
\usepackage{iftex}
\ifPDFTeX
  \usepackage[T1]{fontenc}
  \usepackage[utf8]{inputenc}
  \usepackage{textcomp} % provide euro and other symbols
\else % if luatex or xetex
  \usepackage{unicode-math}
  \defaultfontfeatures{Scale=MatchLowercase}
  \defaultfontfeatures[\rmfamily]{Ligatures=TeX,Scale=1}
\fi
\usefonttheme{structurebold}
% Use upquote if available, for straight quotes in verbatim environments
\IfFileExists{upquote.sty}{\usepackage{upquote}}{}
\IfFileExists{microtype.sty}{% use microtype if available
  \usepackage[]{microtype}
  \UseMicrotypeSet[protrusion]{basicmath} % disable protrusion for tt fonts
}{}
\makeatletter
\@ifundefined{KOMAClassName}{% if non-KOMA class
  \IfFileExists{parskip.sty}{%
    \usepackage{parskip}
  }{% else
    \setlength{\parindent}{0pt}
    \setlength{\parskip}{6pt plus 2pt minus 1pt}}
}{% if KOMA class
  \KOMAoptions{parskip=half}}
\makeatother
\usepackage{xcolor}
\newif\ifbibliography
\usepackage{longtable,booktabs,array}
\usepackage{calc} % for calculating minipage widths
\usepackage{caption}
% Make caption package work with longtable
\makeatletter
\def\fnum@table{\tablename~\thetable}
\makeatother
\usepackage{graphicx}
\makeatletter
\def\maxwidth{\ifdim\Gin@nat@width>\linewidth\linewidth\else\Gin@nat@width\fi}
\def\maxheight{\ifdim\Gin@nat@height>\textheight\textheight\else\Gin@nat@height\fi}
\makeatother
% Scale images if necessary, so that they will not overflow the page
% margins by default, and it is still possible to overwrite the defaults
% using explicit options in \includegraphics[width, height, ...]{}
\setkeys{Gin}{width=\maxwidth,height=\maxheight,keepaspectratio}
% Set default figure placement to htbp
\makeatletter
\def\fps@figure{htbp}
\makeatother
\setlength{\emergencystretch}{3em} % prevent overfull lines
\providecommand{\tightlist}{%
  \setlength{\itemsep}{0pt}\setlength{\parskip}{0pt}}
\setcounter{secnumdepth}{5}
\newlength{\cslhangindent}
\setlength{\cslhangindent}{1.5em}
\newlength{\csllabelwidth}
\setlength{\csllabelwidth}{3em}
\newlength{\cslentryspacingunit} % times entry-spacing
\setlength{\cslentryspacingunit}{\parskip}
\newenvironment{CSLReferences}[2] % #1 hanging-ident, #2 entry spacing
 {% don't indent paragraphs
  \setlength{\parindent}{0pt}
  % turn on hanging indent if param 1 is 1
  \ifodd #1
  \let\oldpar\par
  \def\par{\hangindent=\cslhangindent\oldpar}
  \fi
  % set entry spacing
  \setlength{\parskip}{#2\cslentryspacingunit}
 }%
 {}
\usepackage{calc}
\newcommand{\CSLBlock}[1]{#1\hfill\break}
\newcommand{\CSLLeftMargin}[1]{\parbox[t]{\csllabelwidth}{#1}}
\newcommand{\CSLRightInline}[1]{\parbox[t]{\linewidth - \csllabelwidth}{#1}\break}
\newcommand{\CSLIndent}[1]{\hspace{\cslhangindent}#1}
\ifLuaTeX
  \usepackage{selnolig}  % disable illegal ligatures
\fi
\IfFileExists{bookmark.sty}{\usepackage{bookmark}}{\usepackage{hyperref}}
\IfFileExists{xurl.sty}{\usepackage{xurl}}{} % add URL line breaks if available
\urlstyle{same} % disable monospaced font for URLs
\hypersetup{
  pdftitle={Pre-Trained Denoising Autoencoders Long Short-Term Memory Networks as probabilistic Models for Estimation of Distribution Genetic Programming},
  pdfauthor={Roman Hoehn},
  hidelinks,
  pdfcreator={LaTeX via pandoc}}

\title{Pre-Trained Denoising Autoencoders Long Short-Term Memory
Networks as probabilistic Models for Estimation of Distribution Genetic
Programming}
\subtitle{Kolloquium zur Masterarbeit im M.Sc. Wirtschaftspädagogik}
\author{Roman Hoehn}
\date{Datum: 25.01.2023}
\institute{Johannes Gutenberg-Universität Mainz}

\begin{document}
\frame{\titlepage}

\begin{frame}[allowframebreaks]
  \tableofcontents[hideallsubsections]
\end{frame}
\hypertarget{einleitung}{%
\section{Einleitung}\label{einleitung}}

\begin{frame}{Forschungsfrage}
\protect\hypertarget{forschungsfrage}{}
\emph{Kann das Suchverhalten der Denoising Autoencoder Genetic
Programming (DAE-GP) Metaheuristik durch den Einsatz einer Pre-Training
Strategie optimiert werden?}

Welchen Effekt hat Pre-Training auf:

\begin{enumerate}
\tightlist
\item
  das Generalisierungsverhalten von DAE-GP?
\item
  die Qualität der gefundenen Programme (Fitness/Programmlänge)?
\item
  die Populationsdiversität?
\item
  das Laufzeitverhalten?
\end{enumerate}

Anwendungsgebiet: Symbolische Regression, insbesondere am Aifoil
Datensatz
\end{frame}

\hypertarget{denoising-autoencoder-genetic-programming}{%
\section{Denoising Autoencoder Genetic
Programming}\label{denoising-autoencoder-genetic-programming}}

\begin{frame}{Übersicht}
\protect\hypertarget{uxfcbersicht}{}
\begin{itemize}
\tightlist
\item
  Metaheuristik basierend auf genetischer Programmierung (GP)
\item
  Ersetzung der Variationsoperatoren von GP durch künstliche, neuronale
  Netze zur Optimierung des Suchverhaltens\footnote<.->{Wittenberg,
    Rothlauf and Schweim
    (\protect\hyperlink{ref-dae-gp_2020_rtree}{2020})}
\item
  Variante des Estimation of Distribution-GP (EDA-GP)
\item
  Einsatz von Pre-Training in mehreren Publikationen als möglicher Weg
  für eine weitere Optimierung vorgeschlagen\footnote<.->{Wittenberg and
    Rothlauf (\protect\hyperlink{ref-dae-gp_2022_symreg}{2022})}
  \footnote<.->{Wittenberg
    (\protect\hyperlink{ref-daegp_explore_exploit}{2022})}
\end{itemize}
\end{frame}

\begin{frame}{Estimation of Distribution Algorithmen (EDA)}
\protect\hypertarget{estimation-of-distribution-algorithmen-eda}{}
\begin{itemize}
\tightlist
\item
  Entwicklung neuer Rekombinationsoperatoren für evolutionäre
  Algorithmen basierend auf dem Einsatz von probabilistischen
  Modellen\footnote<.->{Rothlauf
    (\protect\hyperlink{ref-design_of_modern_heuristics}{2011})}
\item
  Hypothese: Problemspezifische Abhängigkeiten zwischen
  Entscheidungsvariablen können bei der Erzeugung neuer Individuen
  besser berücksichtigt werden als bei traditionellen
  Rekombinationsoperatoren\footnote<.->{Mühlenbein and Paass
    (\protect\hyperlink{ref-edaOrig1996}{1996})}
\item
  Weitere Verbreitung im Bereich der genetischen Algorithmen (GA) als
  für GP
\end{itemize}
\end{frame}

\begin{frame}{Denoising Autoencoder Estimation of Distribution
Algorithmen (DAE-EDA)}
\protect\hypertarget{denoising-autoencoder-estimation-of-distribution-algorithmen-dae-eda}{}
Idee: Einsatz von Denoising Autoencoders\footnote<.->{Vincent \emph{et
  al.} (\protect\hyperlink{ref-dae_orig2008}{2008})} (DAE) als
probabilistisches Modell für genetische Algorithmen \footnote<.->{Probst
  and Rothlauf (\protect\hyperlink{ref-harmless_overfitting_eda}{2020})}

2 Phasen Ansatz:

\begin{enumerate}
\tightlist
\item
  Model Building: Modell ``lernt'' die Eigenschaften von ausgewählten
  Lösungen hoher Güte durch das Trainieren eines DAEs
\item
  Model Sampling: Neue Lösungen werden erzeugt durch das propagieren von
  bestehenden, mutierten Lösungen durch das erlernte Modell
\end{enumerate}
\end{frame}

\begin{frame}{Denoising Autoencoder Genetic Programming (DAE-GP)}
\protect\hypertarget{denoising-autoencoder-genetic-programming-dae-gp}{}
\begin{itemize}
\tightlist
\item
  Adaptierung des DAE-EDA Algorithmus auf GP
\item
  Darstellung von Individuen als Zeichenketten in prefix Notation
\item
  seq2seq learning Problem: Einsatz von DAE - Long Short Term Memory
  Netzwerken (DAE-LSTM)
\end{itemize}
\end{frame}

\begin{frame}{DAE-GP Ablauf}
\protect\hypertarget{dae-gp-ablauf}{}
\includegraphics[width=\textwidth,height=0.8\textheight]{./img/flowcharts/dae-gp.png}
\end{frame}

\begin{frame}{Pre-Training}
\protect\hypertarget{pre-training}{}
Idee: Modelle werden vor ihrem eigentlichen Einsatz zum Lösen eines
Problems auf möglichst großen Datensätzen vortrainiert

Mögliche Vorteile durch Pre-Training\footnote<.->{Erhan \emph{et al.}
  (\protect\hyperlink{ref-pmlr-v5-erhan09a}{2009})}:

\begin{enumerate}
\tightlist
\item
  Geringere Bedarf an Trainings Daten für vortrainierte Modelle
\item
  Reduktion der Trainingszeiten/Laufzeiten
\item
  Verbesserung der Güte des Modells
\item
  Verbessertes Generalisierungsverhalten des Modells
\end{enumerate}
\end{frame}

\hypertarget{aktueller-forschungsstand}{%
\section{Aktueller Forschungsstand}\label{aktueller-forschungsstand}}

\begin{frame}{Generalisiertes Royal Tree Problem\footnote<.->{Wittenberg,
  Rothlauf and Schweim (\protect\hyperlink{ref-dae-gp_2020_rtree}{2020})}
(GRT):}
\protect\hypertarget{generalisiertes-royal-tree-problem-grt}{}
\begin{itemize}
\tightlist
\item
  Einfaches Suchproblem mit hoher Lokalität
\item
  DAE-GP erzeugt durch Model Sampling Lösungskandidaten mit höherer
  Fitness als GP
\item
  Hohe Güte der erzeugten Lösungskandidaten resultiert in besserer
  Performance
\item
  Perfomance Vorteil steigt mit zunehmender Komplexität des GRT Problems
\end{itemize}
\end{frame}

\begin{frame}{Symbolische Regression\footnote<.->{Wittenberg and
  Rothlauf (\protect\hyperlink{ref-dae-gp_2022_symreg}{2022})}:}
\protect\hypertarget{symbolische-regression}{}
\begin{itemize}
\item
  Airfoil Datensatz für Real-World symbolische Regression
\item
  DAE-GP erzeugt für eine vorgegebene Anzahl von Fitness Evaluationen im
  Vergleich zu GP:

  \begin{enumerate}
  \tightlist
  \item
    Lösungen mit höherer Fitness
  \item
    Lösungen mit geringerer Größe
  \end{enumerate}
\end{itemize}
\end{frame}

\begin{frame}{Pre-Training für Denoising Autoencoders\footnote<.->{Erhan
  \emph{et al.} (\protect\hyperlink{ref-pmlr-v5-erhan09a}{2009})}}
\protect\hypertarget{pre-training-fuxfcr-denoising-autoencoders}{}
\begin{block}{Positive Wirkung von Pre-Training auf DAE}
\protect\hypertarget{positive-wirkung-von-pre-training-auf-dae}{}
\begin{enumerate}
\tightlist
\item
  Gesteigerte Modell Performance (sinkender Testfehler)
\item
  Besserer Generalisierungsfähigkeit
\item
  Erhöhter Robustness des Algorithmus (sinkende Varianz des Testfehlers)
\end{enumerate}
\end{block}

\begin{block}{Einfluss der Modell Architektur}
\protect\hypertarget{einfluss-der-modell-architektur}{}
\begin{itemize}
\tightlist
\item
  Positiver Effekt steigt mit zunehmender Komplexitität des DAE
\item
  Je mehr versteckte Layer oder Neuronen pro verstecktem Layer vorhanden
  sind, desto besserer Effekt des Pre-Trainings
\item
  Für sehr kleine DAE, zeigt Pre-Training jedoch inverse, negative
  Auswirkung auf die Modell Performance
\end{itemize}
\end{block}
\end{frame}

\hypertarget{implementation}{%
\section{Implementation}\label{implementation}}

\begin{frame}{Überblick}
\protect\hypertarget{uxfcberblick}{}
Gewählte Pre-Training Strategie:

\begin{itemize}
\item
  (Klassisches) Pre-Training: Einmaliges Pre-Training eines DAE-LSTM mit
  einer großen Population aus Lösungskandidaten \(\hat{P}_{train}\)
  (ramped Half and Half Initialisierung)
\item
  Trainingsmethode Early Stopping: Abbruch des Trainings sobald der
  Testfehler für eine seperate Population \(\hat{P}_{test}\) konvergiert
\end{itemize}

\emph{Ausschluss weiterer Pre-Training Strategien wie Re-Use Learning,
Few-Shot Learning}
\end{frame}

\begin{frame}{Pre-Trained DAE-GP Ablauf}
\protect\hypertarget{pre-trained-dae-gp-ablauf}{}
\begin{center}\includegraphics[width=14.28in]{./img/flowcharts/pt-dae-gp} \end{center}
\end{frame}

\begin{frame}{Herausforderungen}
\protect\hypertarget{herausforderungen}{}
\ldots{} hidden neurons \ldots{}
\end{frame}

\hypertarget{untersuchung-des-generalisierungsverhalten}{%
\section{Untersuchung des
Generalisierungsverhalten}\label{untersuchung-des-generalisierungsverhalten}}

\begin{frame}{Fragestellung}
\protect\hypertarget{fragestellung}{}
Welchen Einfluss hat die Dimension der eingesetzten DAE-LSTMs auf das
Generalisierungsverhalten von DAE-GP in Kombination mit Pre-Training?

Experimenteller Aufbau:

\begin{itemize}
\tightlist
\item
  Ausschließliche Betrachtung des Rekonstruktionsfehlers der ersten
  Generation (Nutzung von seperaten Test- und Trainingspopulationen)
\item
  DAE-LSTM Training erfolgt über eine fixe Anzahl von \(1000\) Epochen
  (kein frühzeitiger Abbruch bei Konvergenz des Testfehlers)
\end{itemize}

Insgesamt 12 Subexperimente mit jeweils 10 Durchläufen (\(120\)
Gesamtdurchläufe):

\begin{itemize}
\tightlist
\item
  Variable Anzahl von Hidden Neurons (50, 100, 200) mit einem Hidden
  Layer
\item
  Variable Anzahl von Hidden Layers (1, 2, 3) mit 100 Hidden Neurons
\item
  (normales) DAE-GP und pre-trained DAE-GP
\end{itemize}
\end{frame}

\begin{frame}{Einfluss der Anzahl von Hidden Layers (1/2)}
\protect\hypertarget{einfluss-der-anzahl-von-hidden-layers-12}{}
\includegraphics[width=\textwidth,height=0.85\textheight]{./img/airfoil_firstGen/airfoil_firstGen_median_training_error_by_layers_6plots.png}
\end{frame}

\begin{frame}{Einfluss der Anzahl von Hidden Layers (2/2)}
\protect\hypertarget{einfluss-der-anzahl-von-hidden-layers-22}{}
\includegraphics[width=\textwidth,height=0.85\textheight]{./img/airfoil_firstGen/airfoil_firstGen_median_training_error_by_layers_3plots.png}
\end{frame}

\begin{frame}{Einfluss der Anzahl von Hidden Neurons (1/2)}
\protect\hypertarget{einfluss-der-anzahl-von-hidden-neurons-12}{}
\includegraphics[width=\textwidth,height=0.85\textheight]{./img/airfoil_firstGen/airfoil_firstGen_median_training_error_by_neurons_6plots.png}
\end{frame}

\begin{frame}{Einfluss der Anzahl von Hidden Neurons (2/2)}
\protect\hypertarget{einfluss-der-anzahl-von-hidden-neurons-22}{}
\includegraphics[width=\textwidth,height=0.85\textheight]{./img/airfoil_firstGen/airfoil_firstGen_median_training_error_by_neurons_6plots.png}
\end{frame}

\begin{frame}{Fazit Generalisierungsverhalten}
\protect\hypertarget{fazit-generalisierungsverhalten}{}
\ldots{}
\end{frame}

\hypertarget{untersuchung-des-suchverhaltens-bei-symbolischer-regression}{%
\section{Untersuchung des Suchverhaltens bei symbolischer
Regression}\label{untersuchung-des-suchverhaltens-bei-symbolischer-regression}}

\begin{frame}{Fragestellung}
\protect\hypertarget{fragestellung-1}{}
Welchen Einfluss hat die Verwendung einer Pre-Training Strategie auf das
Suchverhalten von DAE-GP bei der Anwendung auf symbolische
Regressionsprobleme?

Aufbau: Betrachtung des Suchverhalten über je \(10\) Gesamtduchläufe für
DAE-GP und pre-trained DAE-GP

Fokus insbesondere auf:

\begin{itemize}
\tightlist
\item
  Lösungsqualität (Fitness)
\item
  Größe der gefundenen Lösungen (Anzahl an Knoten)
\item
  Populationsdiversität
\item
  Anzahl an Trainingsepochen pro Generation
\end{itemize}
\end{frame}

\begin{frame}{Hyperparameter}
\protect\hypertarget{hyperparameter}{}
\begin{table}
\centering\begingroup\fontsize{7}{9}\selectfont

\resizebox{\linewidth}{!}{
\begin{tabular}{l|l}
\hline
\textbf{parameter} & \textbf{value}\\
\hline
populationSize & 500\\
\hline
generations & 30\\
\hline
fitness & RMSE\\
\hline
TrainingMode & Convergence\\
\hline
SamplingSteps & 2\\
\hline
hiddenLayers & 2\\
\hline
Selection & Binary Tournament Selection\\
\hline
Pre-Training PopulationSize(Training/Validation) & 10000/100000\\
\hline
Pre-Training TrainingMode & Early Stopping\\
\hline
\end{tabular}}
\endgroup{}
\end{table}
\end{frame}

\begin{frame}{Funktions Set Symbolische Regression}
\protect\hypertarget{funktions-set-symbolische-regression}{}
\begin{table}[!h]
\centering
\begin{tabular}{l|r}
\hline
\textbf{function.} & \textbf{arity}\\
\hline
addition & 2\\
\hline
subtraction & 2\\
\hline
multiplication & 2\\
\hline
analytic\_quotient & 2\\
\hline
\end{tabular}
\end{table}
\end{frame}

\begin{frame}{Airfoil - Entwicklung der Fitness über Generationen}
\protect\hypertarget{airfoil---entwicklung-der-fitness-uxfcber-generationen}{}
\includegraphics[width=\textwidth,height=0.9\textheight]{~/masterThesis/master_thesis/img/airfoil_2hl_maxIndSize_fullRun_30gens_withGP/mean_median_fitness_byGens.png}
\end{frame}

\begin{frame}{Airfoil - Verteilung der finalen Fitness}
\protect\hypertarget{airfoil---verteilung-der-finalen-fitness}{}
\includegraphics[width=\textwidth,height=0.9\textheight]{~/masterThesis/master_thesis/img/airfoil_2hl_maxIndSize_fullRun_30gens/final_fit_boxplot.png}
\end{frame}

\begin{frame}{Airfoil - Größe der besten Lösungskandidaten}
\protect\hypertarget{airfoil---gruxf6uxdfe-der-besten-luxf6sungskandidaten}{}
\includegraphics{~/masterThesis/master_thesis/img/airfoil_2hl_maxIndSize_fullRun_30gens_withGP/mean_Size_byGens.png}
\end{frame}

\begin{frame}{Kontrollexperiment: Reduzierung der DAE-LSTM Dimension für
Airfoil Datensatz}
\protect\hypertarget{kontrollexperiment-reduzierung-der-dae-lstm-dimension-fuxfcr-airfoil-datensatz}{}
Frage: Welchen Einfluss hat die Reduktion der DAE-LSTM auf 1 hidden
Layer (HL)?
\end{frame}

\begin{frame}{Airfoil - Entwicklung der Fitness (1 HL)}
\protect\hypertarget{airfoil---entwicklung-der-fitness-1-hl}{}
\includegraphics{~/masterThesis/master_thesis/img/airfoil_1hl_maxIndSize_fullRun_30gens/mean_median_fitness_byGens.png}
\end{frame}

\begin{frame}{Airfoil - Verteilung der finalen Fitness (1HL)}
\protect\hypertarget{airfoil---verteilung-der-finalen-fitness-1hl}{}
\includegraphics[width=\textwidth,height=0.75\textheight]{~/masterThesis/master_thesis/img/airfoil_1hl_maxIndSize_fullRun_30gens/final_fit_boxplot.png}
\end{frame}

\begin{frame}{Airfoil - Größe der besten Lösungskandidaten(1HL)}
\protect\hypertarget{airfoil---gruxf6uxdfe-der-besten-luxf6sungskandidaten1hl}{}
\includegraphics[width=\textwidth,height=0.9\textheight]{~/masterThesis/master_thesis/img/airfoil_1hl_maxIndSize_fullRun_30gens/mean_Size_byGens.png}
\end{frame}

\begin{frame}{Anwendung auf weitere Datensätzen}
\protect\hypertarget{anwendung-auf-weitere-datensuxe4tzen}{}
Airfoil Datensatz: Ergebnis deuten bei einer ausreichenden Anzahl von
Hidden Layern auf einen postitiven Effekt der Pre-Training Strategie
hin:

\begin{itemize}
\tightlist
\item
  Höhere Fitness der gefundenen Lösungen
\item
  Kleinere Größe der gefundenen Lösungen
\end{itemize}

Daher: Ausweitung des Experiments auf weitere Datensätze
\end{frame}

\begin{frame}{Übersicht Datensätze}
\protect\hypertarget{uxfcbersicht-datensuxe4tze}{}
\begin{table}[!h]
\centering
\begin{tabular}{l|r|r}
\hline
\textbf{Problem} & \textbf{Observations} & \textbf{Features}\\
\hline
Airfoil & 1503 & 5\\
\hline
Boston\_Housing & 506 & 13\\
\hline
Energy\_Cooling & 768 & 8\\
\hline
Concrete & 1030 & 8\\
\hline
\end{tabular}
\end{table}
\end{frame}

\begin{frame}{Übersicht Median Fitness - Symbolische Regression}
\protect\hypertarget{uxfcbersicht-median-fitness---symbolische-regression}{}
\includegraphics[width=\textwidth,height=0.8\textheight]{~/masterThesis/master_thesis/img/symbolicRegressionSummarized/median_fitness_byGens.png}
\end{frame}

\begin{frame}{Auswertung Fitness (Median) - Symbolische Regression}
\protect\hypertarget{auswertung-fitness-median---symbolische-regression}{}
\begin{longtable}[]{@{}
  >{\centering\arraybackslash}p{(\columnwidth - 12\tabcolsep) * \real{0.1977}}
  >{\centering\arraybackslash}p{(\columnwidth - 12\tabcolsep) * \real{0.1744}}
  >{\centering\arraybackslash}p{(\columnwidth - 12\tabcolsep) * \real{0.0814}}
  >{\centering\arraybackslash}p{(\columnwidth - 12\tabcolsep) * \real{0.1279}}
  >{\centering\arraybackslash}p{(\columnwidth - 12\tabcolsep) * \real{0.1512}}
  >{\centering\arraybackslash}p{(\columnwidth - 12\tabcolsep) * \real{0.1047}}
  >{\centering\arraybackslash}p{(\columnwidth - 12\tabcolsep) * \real{0.1628}}@{}}
\toprule()
\begin{minipage}[b]{\linewidth}\centering
Problem
\end{minipage} & \begin{minipage}[b]{\linewidth}\centering
Hidden-Layers
\end{minipage} & \begin{minipage}[b]{\linewidth}\centering
Set
\end{minipage} & \begin{minipage}[b]{\linewidth}\centering
DAE-GP
\end{minipage} & \begin{minipage}[b]{\linewidth}\centering
Pre-Trained
\end{minipage} & \begin{minipage}[b]{\linewidth}\centering
P-Value
\end{minipage} & \begin{minipage}[b]{\linewidth}\centering
Cliffs-Delta
\end{minipage} \\
\midrule()
\endhead
Airfoil & 1 & Train & \textbf{33.55} & 35.11 & 0.02** & 0.64 \\
& 1 & Test & \textbf{34.03} & 35.81 & 0.14 & 0.40 \\
Airfoil & 2 & Train & 11.64 & \textbf{8.24} & 0.31 & -0.28 \\
& 2 & Test & 11.34 & \textbf{8.06} & 0.57 & -0.16 \\
Boston\_Housing & 2 & Train & 8.23 & \textbf{7.84} & 0.29 & -0.29 \\
& 2 & Test & 8.04 & \textbf{8} & 0.71 & -0.11 \\
Energy(Cooling) & 2 & Train & \textbf{4.42} & 4.65 & 0.02** & 0.63 \\
& 2 & Test & \textbf{4.61} & 4.86 & 0.29 & 0.29 \\
Concrete & 2 & Train & \textbf{16.58} & 16.83 & 0.97 & 0.02 \\
& 2 & Test & \textbf{16.62} & 16.64 & 0.52 & 0.18 \\
\bottomrule()
\end{longtable}
\end{frame}

\begin{frame}{Zusammenfassung Fitness - Symbolische Regression}
\protect\hypertarget{zusammenfassung-fitness---symbolische-regression}{}
Keine Evidenz für einen statistisch signifikanten Einfluss von
Pre-Training auf die erzielte Lösungsqualität in den durchgeführten
Experimenten!

\begin{itemize}
\tightlist
\item
  Airfoil:
\item
  Energy Cooling:
\item
  Boston Housing:
\item
  Concrete:
\end{itemize}
\end{frame}

\begin{frame}{Übersicht Lösungsgröße (Median) - Symbolische Regression}
\protect\hypertarget{uxfcbersicht-luxf6sungsgruxf6uxdfe-median---symbolische-regression}{}
\includegraphics[width=\textwidth,height=0.8\textheight]{~/masterThesis/master_thesis/img/symbolicRegressionSummarized/median_bestSol_size_byGens.png}
\end{frame}

\begin{frame}{Auswertung Lösungsgröße (Median) - Symbolische Regression}
\protect\hypertarget{auswertung-luxf6sungsgruxf6uxdfe-median---symbolische-regression}{}
\begin{longtable}[]{@{}
  >{\centering\arraybackslash}p{(\columnwidth - 10\tabcolsep) * \real{0.2237}}
  >{\centering\arraybackslash}p{(\columnwidth - 10\tabcolsep) * \real{0.1579}}
  >{\centering\arraybackslash}p{(\columnwidth - 10\tabcolsep) * \real{0.1447}}
  >{\centering\arraybackslash}p{(\columnwidth - 10\tabcolsep) * \real{0.1711}}
  >{\centering\arraybackslash}p{(\columnwidth - 10\tabcolsep) * \real{0.1184}}
  >{\centering\arraybackslash}p{(\columnwidth - 10\tabcolsep) * \real{0.1842}}@{}}
\toprule()
\begin{minipage}[b]{\linewidth}\centering
Problem
\end{minipage} & \begin{minipage}[b]{\linewidth}\centering
Hid.Layers
\end{minipage} & \begin{minipage}[b]{\linewidth}\centering
DAE-GP
\end{minipage} & \begin{minipage}[b]{\linewidth}\centering
Pre-Trained
\end{minipage} & \begin{minipage}[b]{\linewidth}\centering
P-Value
\end{minipage} & \begin{minipage}[b]{\linewidth}\centering
Cliffs-Delta
\end{minipage} \\
\midrule()
\endhead
Airfoil & 1 & \textbf{33.55} & 35.11 & 0.02** & 0.64 \\
Airfoil & 2 & 11.64 & \textbf{8.24} & 0.31 & -0.28 \\
Boston\_Housing & 2 & 8.23 & \textbf{7.84} & 0.29 & -0.29 \\
Energy(Cooling) & 2 & \textbf{4.42} & 4.65 & 0.02** & 0.63 \\
Concrete & 2 & \textbf{16.58} & 16.83 & 0.97 & 0.02 \\
\bottomrule()
\end{longtable}
\end{frame}

\begin{frame}{Übersicht Populationsdiversität (Median) - Symbolische
Regression}
\protect\hypertarget{uxfcbersicht-populationsdiversituxe4t-median---symbolische-regression}{}
\includegraphics[width=\textwidth,height=0.8\textheight]{~/masterThesis/master_thesis/img/symbolicRegressionSummarized/median_levDistance_byGen.png}
\end{frame}

\begin{frame}{Auswertung Populationsdiversität (Median) - Symbolische
Regression}
\protect\hypertarget{auswertung-populationsdiversituxe4t-median---symbolische-regression}{}
\begin{longtable}[]{@{}
  >{\centering\arraybackslash}p{(\columnwidth - 10\tabcolsep) * \real{0.2267}}
  >{\centering\arraybackslash}p{(\columnwidth - 10\tabcolsep) * \real{0.1600}}
  >{\centering\arraybackslash}p{(\columnwidth - 10\tabcolsep) * \real{0.1333}}
  >{\centering\arraybackslash}p{(\columnwidth - 10\tabcolsep) * \real{0.1733}}
  >{\centering\arraybackslash}p{(\columnwidth - 10\tabcolsep) * \real{0.1200}}
  >{\centering\arraybackslash}p{(\columnwidth - 10\tabcolsep) * \real{0.1867}}@{}}
\toprule()
\begin{minipage}[b]{\linewidth}\centering
Problem
\end{minipage} & \begin{minipage}[b]{\linewidth}\centering
Hid.Layers
\end{minipage} & \begin{minipage}[b]{\linewidth}\centering
DAE-GP
\end{minipage} & \begin{minipage}[b]{\linewidth}\centering
Pre-Trained
\end{minipage} & \begin{minipage}[b]{\linewidth}\centering
P-Value
\end{minipage} & \begin{minipage}[b]{\linewidth}\centering
Cliffs-Delta
\end{minipage} \\
\midrule()
\endhead
Airfoil & 1 & \textbf{0.34} & 0.49 & 0.02** & 0.36 \\
Airfoil & 2 & \textbf{0.91} & 0.93 & 0.00*** & 0.88 \\
Boston\_Housing & 2 & \textbf{0.94} & 0.95 & 0.00*** & 0.82 \\
Energy(Cooling) & 2 & \textbf{0.93} & 0.94 & 0.00*** & 0.80 \\
Concrete & 2 & \textbf{0.93} & 0.94 & 0.00*** & 0.88 \\
\bottomrule()
\end{longtable}
\end{frame}

\begin{frame}{Übersicht Trainingsepochen pro Generation (Median) -
Symbolische Regression}
\protect\hypertarget{uxfcbersicht-trainingsepochen-pro-generation-median---symbolische-regression}{}
\includegraphics[width=\textwidth,height=0.8\textheight]{~/masterThesis/master_thesis/img/symbolicRegressionSummarized/median_epochsTrained.png}
\end{frame}

\begin{frame}{Auswertung Trainingsepochen pro Generation (Median) -
Symbolische Regression}
\protect\hypertarget{auswertung-trainingsepochen-pro-generation-median---symbolische-regression}{}
\begin{longtable}[]{@{}
  >{\centering\arraybackslash}p{(\columnwidth - 10\tabcolsep) * \real{0.2222}}
  >{\centering\arraybackslash}p{(\columnwidth - 10\tabcolsep) * \real{0.1667}}
  >{\centering\arraybackslash}p{(\columnwidth - 10\tabcolsep) * \real{0.1111}}
  >{\centering\arraybackslash}p{(\columnwidth - 10\tabcolsep) * \real{0.1806}}
  >{\centering\arraybackslash}p{(\columnwidth - 10\tabcolsep) * \real{0.1250}}
  >{\centering\arraybackslash}p{(\columnwidth - 10\tabcolsep) * \real{0.1944}}@{}}
\toprule()
\begin{minipage}[b]{\linewidth}\centering
Problem
\end{minipage} & \begin{minipage}[b]{\linewidth}\centering
Hid.Layers
\end{minipage} & \begin{minipage}[b]{\linewidth}\centering
DAE-GP
\end{minipage} & \begin{minipage}[b]{\linewidth}\centering
Pre-Trained
\end{minipage} & \begin{minipage}[b]{\linewidth}\centering
P-Value
\end{minipage} & \begin{minipage}[b]{\linewidth}\centering
Cliffs-Delta
\end{minipage} \\
\midrule()
\endhead
Airfoil & 1 & 190.93 & \textbf{60} & 0.00*** & -0.94 \\
Airfoil & 2 & 151.87 & \textbf{64.95} & 0.00*** & -0.88 \\
Boston\_Housing & 2 & 182.84 & \textbf{67.27} & 0.00*** & -0.88 \\
Energy\_Cooling & 2 & 169.39 & \textbf{69.87} & 0.00*** & -0.90 \\
concrete & 2 & 171.69 & \textbf{64.9} & 0.00*** & -0.92 \\
\bottomrule()
\end{longtable}
\end{frame}

\hypertarget{weitere-verfolgte-ansuxe4tze}{%
\section{Weitere verfolgte Ansätze}\label{weitere-verfolgte-ansuxe4tze}}

\begin{frame}{Eingeschobenes Pre-Training in der 2. Generation}
\protect\hypertarget{eingeschobenes-pre-training-in-der-2.-generation}{}
Ansatz: Deutliche Fitnessverbesserung finden oft bereits in der ersten
Generationen der evolutionären Suche statt

\begin{itemize}
\tightlist
\item
  Sampling des DAE-LSTMs der ersten Generation \(M_1\) zur
  Initialisierung der Pre-Training Population \(\hat{P}\)
\item
  Training des pre-training modells \(\hat{M}\) mit \(\hat{P}\) nach
  Abschluss der ersten Generation
\item
  DAE-LSTM Modelle werden ab Generation 2 mit den Parametern von
  \(\hat{M}\) initialisiert
\end{itemize}
\end{frame}

\begin{frame}{Eingeschobenes Pre-Training in der 2. Generation - Airfoil
Datensatz}
\protect\hypertarget{eingeschobenes-pre-training-in-der-2.-generation---airfoil-datensatz}{}
\includegraphics[width=\textwidth,height=0.8\textheight]{~/masterThesis/master_thesis/img/2ndGenPT_airfoil_2hl_150hn_fullRun/median_fitness_byGens.png}
\end{frame}

\begin{frame}{Grow Initialisierung der Pre-Training Population}
\protect\hypertarget{grow-initialisierung-der-pre-training-population}{}
\end{frame}

\hypertarget{references}{%
\section*{References}\label{references}}
\addcontentsline{toc}{section}{References}

\begin{frame}[allowframebreaks]{References}
\hypertarget{refs}{}
\begin{CSLReferences}{0}{0}
\leavevmode\vadjust pre{\hypertarget{ref-pmlr-v5-erhan09a}{}}%
Erhan, D. \emph{et al.} (2009) {`The difficulty of training deep
architectures and the effect of unsupervised pre-training'}, in D. van
Dyk and M. Welling (eds) \emph{Proceedings of the twelth international
conference on artificial intelligence and statistics}. Hilton Clearwater
Beach Resort, Clearwater Beach, Florida USA: PMLR (Proceedings of
machine learning research), pp. 153--160. Available at:
\url{https://proceedings.mlr.press/v5/erhan09a.html}.

\leavevmode\vadjust pre{\hypertarget{ref-edaOrig1996}{}}%
Mühlenbein, H. and Paass, G. (1996) {`From recombination of genes to the
estimation of distributions i. Binary parameters.'}, in \emph{From
Recombination of Genes to the Estimation of Distributions I. Binary
Parameters}, pp. 178--187.

\leavevmode\vadjust pre{\hypertarget{ref-harmless_overfitting_eda}{}}%
Probst, M. and Rothlauf, F. (2020) {`Harmless overfitting: Using
denoising autoencoders in estimation of distribution algorithms'},
\emph{Journal of Machine Learning Research}, 21(78), pp. 1--31.
Available at: \url{http://jmlr.org/papers/v21/16-543.html}.

\leavevmode\vadjust pre{\hypertarget{ref-design_of_modern_heuristics}{}}%
Rothlauf, F. (2011) \emph{Design of modern heuristics: Principles and
application}, \emph{Natural Computing Series}. Available at:
\url{https://doi.org/10.1007/978-3-540-72962-4}.

\leavevmode\vadjust pre{\hypertarget{ref-dae_orig2008}{}}%
Vincent, P. \emph{et al.} (2008) {`Extracting and composing robust
features with denoising autoencoders'}, in \emph{Proceedings of the 25th
International Conference on Machine Learning}, pp. 1096--1103. Available
at: \url{https://doi.org/10.1145/1390156.1390294}.

\leavevmode\vadjust pre{\hypertarget{ref-daegp_explore_exploit}{}}%
Wittenberg, D. (2022) {`Using denoising autoencoder genetic programming
to~control exploration and~exploitation in~search'}, in E. Medvet, G.
Pappa, and B. Xue (eds) \emph{Genetic programming}. Cham: Springer
International Publishing, pp. 102--117.

\leavevmode\vadjust pre{\hypertarget{ref-dae-gp_2022_symreg}{}}%
Wittenberg, D. and Rothlauf, F. (2022) {`Denoising autoencoder genetic
programming for real-world symbolic regression'}, in \emph{Proceedings
of the genetic and evolutionary computation conference companion}. New
York, NY, USA: Association for Computing Machinery (GECCO '22), pp.
612--614. Available at: \url{https://doi.org/10.1145/3520304.3528921}.

\leavevmode\vadjust pre{\hypertarget{ref-dae-gp_2020_rtree}{}}%
Wittenberg, D., Rothlauf, F. and Schweim, D. (2020) {`DAE-GP: Denoising
autoencoder LSTM networks as probabilistic models in estimation of
distribution genetic programming'}, in \emph{Proceedings of the 2020
genetic and evolutionary computation conference}. New York, NY, USA:
Association for Computing Machinery (GECCO '20), pp. 1037--1045.
Available at: \url{https://doi.org/10.1145/3377930.3390180}.

\end{CSLReferences}
\end{frame}

\end{document}
